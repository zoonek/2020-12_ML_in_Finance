\documentclass[a4paper,12pt,compress,serif]{beamer}
%\documentclass[aspectratio=169,12pt,compress,serif]{beamer}

%%
%% Compile with: 
%%   Rscript -e 'library(knitr);knit("template_beamer.Rnw")'
%%   xelatex --shell-escape --interaction=nonstopmode template_beamer.tex
%%   xelatex --shell-escape --interaction=nonstopmode template_beamer.tex
%%
%% The --shell-escape argument is needed if you use the minted package, to typeset code. 
%% For the table of contents and the references to be correct, you need to compile 2 or 3 times.
%%

\usetheme{Warsaw}
\beamertemplatenavigationsymbolsempty   % Remove the clickable icons at the bottom right
%\useoutertheme{infolines}               % Page numbers

\defbeamertemplate*{footline}{shadow theme}
{%
  \leavevmode%
  \hbox{\begin{beamercolorbox}[wd=.5\paperwidth,ht=2.5ex,dp=1.125ex,leftskip=.3cm plus1fil,rightskip=.3cm]{author in head/foot}%
    \usebeamerfont{author in head/foot}\insertshortauthor\hfill\ 
  \end{beamercolorbox}%
  \begin{beamercolorbox}[wd=.5\paperwidth,ht=2.5ex,dp=1.125ex,leftskip=.3cm,rightskip=.3cm plus1fil]{title in head/foot}%
    \usebeamerfont{title in head/foot}\insertshorttitle\hfill\insertframenumber\,/\,\inserttotalframenumber%
  \end{beamercolorbox}}%
  \vskip0pt%
}

%%
%% Metadata
%%
\title{Deep Learning Factor Models}
\author[Vincent Zoonekynd and Miquel Noguer i Alonso]{Vincent Zoonekynd \\ Miquel Noguer i Alonso} %  \href{mailto:zoonek@gmail.com}{zoonek@gmail.com}}

%%
%% Fonts
%%
\usepackage[cm-default,no-math]{fontspec}
\usepackage{metalogo}                      % \XeLaTeX  (put this before loading the other fonts)
\def\Japanese{\fontspec{MS Mincho}}        % To insert Japanese text: {\Japanese 日本語}
%%
%% Uncomment one of the following sections:
%%
%% 1. Default: Computer Modern, Sans Serif
%% (nothing to do)
%%
%% 2. Concrete with Euler
%% Also needs: \documentclass[serif]{beamer}
%%
\usepackage[T1]{fontenc}      
\usepackage{ccfonts,eulervm}  
%\usepackage{concrete}
%%
%% 3. Arev (based on Bitstream Vera), another sans-serif fonts
%%
%\usepackage{arev}  
%%

\usepackage{minted}
%\setsansfont{Calibri}
\setmonofont[Scale=.8]{Consolas}

\makeatletter
% For the disclaimer, use the least readable Sans Serif font we have:
%\newcommand{\disclaimersize}{\fontspec{Univers Deutsche Bank}\@setfontsize\miniscule{3.5}{4}}
% Calibri is even more compact, and has the advantage of having a bold variant:
%\newcommand{\disclaimersize}{\fontspec{Calibri}\@setfontsize\miniscule{3.5}{4}}
%\newcommand{\disclaimersize}{\fontspec{Calibri}\@setfontsize\miniscule{2.5}{2.8}}
\makeatother

%%
%% PDF 
%%
%\usepackage[pdftitle={\SHORTTITLE},pdfauthor={\SHORTAUTHOR},hidelinks]{hyperref}
\usepackage{url}

%%
%% Plots
%%
\usepackage[all]{xy}
\usepackage{tikz,pgflibraryshapes,tikz-cd}
\usetikzlibrary{calc,matrix,arrows,positioning}
\usepackage{pgfplots}
\usepackage{calc}

%%%%%%%%%%%%%%%%%%%%%%%%%%%%%%%%%%%%%%%%%%%%%%%%%%%%%%%%%%%%

%%
%% Mathematics
%%
\usepackage{amsmath,amsfonts,amssymb,mathrsfs,stmaryrd}
\usepackage{accents}
\let\mathbb\mathbf
\def\ZZ{\mathbb{Z}}
\def\NN{\mathbb{N}}
\def\RR{\mathbb{R}}
\def\PP{\mathbb{P}}
\def\QQ{\mathbb{Q}}
\def\CC{\mathbb{C}}
\def\SS{\mathbb{S}}
\def\FF{\mathbb{F}}
\DeclareMathOperator{\sinc}{sinc}
\DeclareMathOperator{\cor}{Cor}
\DeclareMathOperator{\corr}{Corr}
\DeclareMathOperator{\sd}{Sd}
\DeclareMathOperator{\logit}{logit}
\DeclareMathOperator{\supp}{supp}
\let\olddiv\div
\let\div\relax
\DeclareMathOperator{\div}{div}
\let\max\relax
\let\min\relax
\DeclareMathOperator*{\min}{Min}
\DeclareMathOperator*{\max}{Max}
\DeclareMathOperator*{\argmax}{Argmax}
\DeclareMathOperator*{\argmin}{Argmin}
\DeclareMathOperator*{\rank}{rank}
\DeclareMathOperator*{\mean}{Mean}
\DeclareMathOperator*{\Span}{Span}
\def\n#1{\left\|#1\right\|}
\let\leq\leqslant
\let\geq\geqslant
\let\inter\cap
\let\union\cup
\let\tens\otimes
\def\xor{\mathop{\textsc{xor}}}
\let\nothing\varnothing

\newcommand\independent{\protect\mathpalette{\protect\independenT}{\perp}}
\def\independenT#1#2{\mathrel{\rlap{$#1#2$}\mkern4mu{#1#2}}}
\let\indep\independent
%\def\indep{\perp \! \! \! \perp}

\def\bra#1{\left\langle#1\right|}
\def\ket#1{\left|#1\right\rangle}
\def\braket#1#2{\left\langle#1\middle|#2\right\rangle}
\def\braKet#1#2#3{\left\langle#1\middle|#2\middle|#3\right\rangle}

\DeclareMathOperator{\pr}{pr}
\let\max\relax
\DeclareMathOperator*{\max}{Max}
\DeclareMathOperator{\var}{Var}
\DeclareMathOperator{\vect}{Vect}
\DeclareMathOperator{\cov}{Cov}
\DeclareMathOperator{\tr}{tr}
\DeclareMathOperator{\re}{Re}
\DeclareMathOperator{\diag}{diag}
\DeclareMathOperator{\grad}{grad}
\DeclareMathOperator{\pv}{pv}
\def\va#1{\left| #1 \right|}
\let\epsilon\varepsilon
\let\subsetstr\varsubsetneq

%%%%%%%%%%%%%%%%%%%%%%%%%%%%%%%%%%%%%%%%%%%%%%%%%%%%%%%%%%%%

%%
%% Header and footers
%%
\usepackage[en-GB]{datetime2}
\newcommand{\dateseparator}{-}
\newcommand{\todayiso}{\the\year \dateseparator \twodigit\month \dateseparator \twodigit\day}
%\usepackage[level,nodayofweek]{datetime}
%\usepackage{etoolbox}
%\patchcmd{\formatdate}{,}{}{}{}  % http://tex.stackexchange.com/questions/55304/date-format-in-text-and-address-block
\usepackage{xcolor}
%\renewcommand{\dateseparator}{-}
%\newcommand{\todayiso}{\the\year \dateseparator \twodigit\month \dateseparator \twodigit\day}

%%
%% Tables
%%
\usepackage{booktabs,colortbl,xcolor,longtable}
\definecolor{bgred}{HTML}{FBB4AE}   % From Brewer's ``Pastel1'' palette
\definecolor{bgblue}{HTML}{B3CDE3}
\definecolor{bggray}{HTML}{EEEEEE}
\definecolor{myred}{HTML}{AA0000}  % Darker than pure red
\usepackage{sansmath}  % Defines \sansmath, which I use for the tables.
\usepackage{comment}
\usepackage{multirow}

\usepackage{paralist}
\usepackage{xspace}

\usepackage{kantlipsum}
\usepackage{multicol}
\def\columnseprule{.4pt}

% https://tex.stackexchange.com/questions/160825/modifying-margins-for-one-slide
\newcommand\Wider[2][3em]{%
\makebox[\linewidth][c]{%
  \begin{minipage}{\dimexpr\textwidth+#1\relax}
  \raggedright#2
  \end{minipage}%
  }%
}

\begin{document}


%%%%%%%%%%%%%%%%%%%%%%%%%%%%%%%%%%%%%%%%%%%%%%%%%%%%%%%%%%%%

\begin{frame}
  \titlepage
\end{frame}

\begin{frame}
  \frametitle{Outline}
  \tableofcontents
\end{frame}

\section{Motivations}
\begin{frame}
  \frametitle{Factor models}
  \begin{align*}
    y &= \alpha + \beta \text{Market} + \text{noise} \\[1cm]
    y &= \alpha + \sum \beta_i F_i + \text{noise}
  \end{align*}
\end{frame}

\begin{frame}
  \frametitle{Investment strategy}
  \begin{itemize}
  \item Forecast returns
    \[ y = \alpha + \sum_i \beta_i x_i + \epsilon \]
  \item Portfolio optimization
    \[
    \begin{array}{ll}
      \text{Find} & w \\
      \text{To maximize} & w' \hat y - \lambda w' V w \\
      \text{Such that} & w' \mathbf 1 = 1 \\
      & w \geq 0 
    \end{array}
    \]
  \end{itemize}
\end{frame}

\begin{frame}
  \frametitle{Aims}
  Use deep learning to:
  \begin{itemize}
  \item Forecast returns 
  \item Build an investment strategy
  \end{itemize}
\end{frame}

\section{Data}
\frame{\tableofcontents[currentsection]}

\begin{frame}
  \frametitle{Data}
  From the book \href{http://www.mlfactor.com/}{\emph{Machine Learning for Factor Investing} (G. Coqueret and T. Guida)}
  \begin{itemize}
  \item 1200 stocks
  \item Monthly data, from 1998-11 to 2019-03
  \item 100 features: momentum, earnings yield, quality, etc.
  \end{itemize}
\end{frame}

\begin{frame}
  \frametitle{Data preparation}
  Predictors:
  \begin{itemize}
  \item Uniformize
  \end{itemize}

  \bigskip
  Target (1-month forward returns):
  \begin{itemize}
  \item Excess wrt market (or sector)
  \end{itemize}

  \bigskip
  Also try:
  \begin{itemize}
  \item Discretize
  \item Remove outliers
  \item Remove average values
  \item etc.
  \end{itemize}
\end{frame}

\begin{frame}
  \frametitle{Data}
  \centering
  \begin{tabular}{c|c|c|c}
    id & date & forward returns & feature$_1$ \quad $\cdots$ \quad feature$_n$ \\
    \hline
    &&& \\
    &&&
  \end{tabular}
\end{frame}

\begin{frame}
  \frametitle{Data}
\[
\begin{tikzpicture}[>=stealth,yscale=1,xscale=1,inner sep = 2pt, outer sep = 0pt,baseline=-7]
  \draw (0,0) -- (2,0) -- (2,2) -- (0,2) -- cycle;
  %\draw (1,1) -- (3,1) -- (3,3) -- (1,3) -- cycle;
  \draw (3,1) -- (3,3) -- (1,3);
  %\draw (0,0) -- (1,1);
  \draw (0,2) -- (1,3);
  \draw (2,0) -- (3,1);
  \draw (2,2) -- (3,3);
  \draw[ultra thick,red!60!black] (0,2) -- (1,3);
  \draw[ultra thick,blue!70!black] (0,0) -- (2,0);
  \draw[ultra thick,green!50!black] (0,0) -- (0,2);
  \node[anchor=north] at (1,0)    {\color{blue!70!black}date};
  \node[anchor=east]  at (0,1)    {\color{green!50!black}id};
  \node[anchor=east]  at (.6,2.7) {\color{red!60!black}feature};
\end{tikzpicture}
\]
\end{frame}

\section{Models}
\frame{\tableofcontents[currentsection]}

\subsection{Lasso}
\begin{frame}
  \frametitle{Baseline: lasso}
  \[
  \begin{array}{ll}
    \text{Find} & \beta \\
    \text{To minimize} & \n{ y - X \beta ' }_2^2 + \lambda \n{ \beta }
  \end{array}
  \]
\end{frame}

\begin{frame}
  \frametitle{Model interpretation}
  \includegraphics[width=\textwidth,page={1}]{plots/lasso_coefs.pdf}
\end{frame}

\begin{frame}
  \frametitle{Model evaluation}
  \begin{itemize}
  \item As for any supervised learning model
  \item Turn the forecast into an investment strategy
  \end{itemize}
\end{frame}

\begin{frame}
  \frametitle{Baseline: lasso}
  \includegraphics[width=\textwidth,page={1}]{plots/lasso_wealth_20.pdf}
\end{frame}

\begin{frame}
  \frametitle{Baseline: lasso}
  Cross-validation may not work
\end{frame}

\begin{frame}[fragile]
  \frametitle{Lasso}
  \begin{columns}
    \column{\dimexpr\paperwidth-40pt}
\begin{minted}{python}
from sklearn.linear_model import lasso_path
alphas, coef, _ = lasso_path( X = x, y = y, max_iter = 10_000 )
\end{minted}
  \end{columns}
\end{frame}

\subsection{Constrained}
\frame{\tableofcontents[currentsubsection]}
\begin{frame}
  \frametitle{Constrained regression}
  \[
  \begin{array}{ll}
    \text{Find} & \beta \\
    \text{To minimize} & \n{ y - X \beta ' }_2^2 \\
    \text{Such that} & \beta \geq 0
  \end{array}
  \]
\end{frame}

\begin{frame}
  \frametitle{Constrained regression}  
  \includegraphics[width=\textwidth,page={1}]{plots/constrained_regression_wealth.pdf}
\end{frame}

\begin{frame}[fragile]
  \frametitle{Constrained regression}
  \begin{columns}
    \column{\dimexpr\paperwidth-30pt}
\begin{minted}{python}
from scipy.optimize import lsq_linear
r = lsq_linear(
  x, y,
  (
    [ 0      if signs[u] > 0 else -np.inf for u in predictors ],
    [ np.inf if signs[u] > 0 else 0       for u in predictors ],
  )
)    
\end{minted}
  \end{columns}
\end{frame}

\subsection{Non-linear}
\frame{\tableofcontents[currentsubsection]}
\begin{frame}[fragile]
  \frametitle{Nonlinear}
\begin{minted}{python}
class NonLinear(torch.nn.Module):
    def __init__(self,k):
        super(NonLinear,self).__init__()
        self.fc1 = torch.nn.Linear(k,16)
        self.fc2 = torch.nn.Linear(16,4)
        self.fc3 = torch.nn.Linear(4,1)
    def forward(self,x):
        y = self.fc1(x); y = F.relu(y)
        y = self.fc2(y); y = F.relu(y)
        y = self.fc3(y)
        return y
\end{minted}
\end{frame}

\begin{frame}[fragile]
  \frametitle{Nonlinear}
\begin{minted}{python}    
N = 10_000
model = NonLinear(x.shape[1])
criterion = torch.nn.MSELoss()
optimizer = torch.optim.Adam(model.parameters())
for t in tqdm(range(N)):
    y_pred = model(x)
    loss = criterion(y_pred,y)
    optimizer.zero_grad()
    loss.backward()
    optimizer.step()
\end{minted}
\end{frame}

\begin{frame}
  \frametitle{Non-linear model}
  \includegraphics[width=\textwidth,page={1}]{plots/model2_nonlinear_3d_wealth.pdf}
\end{frame}

\begin{frame}[plain]
    \begin{tikzpicture}[remember picture,overlay]
        \node[at=(current page.center)] {
          \includegraphics[width=\paperwidth,page={1}]{plots/model6_nonlinear_scatter_all.png}
        };
    \end{tikzpicture}  
\end{frame}
\begin{frame}[plain]
    \begin{tikzpicture}[remember picture,overlay]
        \node[at=(current page.center)] {
          \includegraphics[width=\paperwidth,page={1}]{plots/model6_nonlinear_hexbin_all.pdf}
        };
    \end{tikzpicture}  
\end{frame}
\begin{frame}[plain]
    \begin{tikzpicture}[remember picture,overlay]
        \node[at=(current page.center)] {
          \includegraphics[width=\paperwidth,page={1}]{plots/model6_nonlinear_copulas_all.pdf}
        };
    \end{tikzpicture}  
\end{frame}
\begin{frame}[plain]
    \begin{tikzpicture}[remember picture,overlay]
        \node[at=(current page.center)] {
          \includegraphics[width=\paperwidth,page={1}]{plots/model6_nonlinear_median_per_quantile_all.pdf}
        };
    \end{tikzpicture}  
\end{frame}

\subsection{End-to-end}
\frame{\tableofcontents[currentsubsection]}
\begin{frame}
  \frametitle{End-to-end model}
  \begin{itemize}
  \item Do not optimize for an intermediate result, return forecasts, but for the final goal, the performance of an investment strategy.
  \item Limit overfitting with minibatches
  \end{itemize}
\end{frame}

\begin{frame}
  \frametitle{End-to-end model}
  \includegraphics[width=\textwidth,page={1}]{plots/model3b_linear_IR_minibatches_wealth.pdf}
\end{frame}

\begin{frame}[fragile]
  \frametitle{End-to-end}
  \begin{columns}
    \column{\dimexpr\paperwidth-10pt}
\begin{minted}{python}
class Linear(torch.nn.Module):
  def __init__(self,k):
    super(Linear,self).__init__()
    self.linear = torch.nn.Linear(k,1)
  def forward(self,xs):
    x, universe = xs
    # x is n×l×k; the linear layer is applied on the last dimension
    y = self.linear(x)   # n×l×1
    p = y.exp()      
    p = p * universe
    p = p[:,:,0]         #  n×l
    p = p / ( 1e-16 + p.sum(axis=0) )  # portolio weights: positive, sum up to 1 for each date
    return p
    
model = Linear(x.shape[2])
optimizer = torch.optim.Adam(model.parameters())
\end{minted}
  \end{columns}
\end{frame}

\begin{frame}[fragile]
  \frametitle{End-to-end}
  \begin{columns}
    \column{\dimexpr\paperwidth-10pt}
\begin{minted}{python}
for t in range(N):
  i = np.random.choice( x.shape[0], x.shape[0] // 2, replace=False ) 
  j = np.random.choice( x.shape[1] - 36 )
  j = np.arange( j, j+36 )    
  w = model( (x[i,:,:][:,j,:], universe[i,:,:][:,j,:]) )
  ratio_returns = w * y[i,:,:][:,j,:][:,:,0].expm1()   # y: forward returns
  ratio_returns = ratio_returns.sum(axis=0)
  log_returns = ratio_returns.log1p()
  IR = log_returns.mean() / log_returns.std()
  loss = -IR
  optimizer.zero_grad()
  loss.backward()
  optimizer.step()
\end{minted}
  \end{columns}
\end{frame}

\begin{frame}
  \frametitle{Non-linear model}
  \includegraphics[width=\textwidth,page={1}]{plots/model6_nonlinear_IR_wealth.pdf}
\end{frame}

\section{Conclusion}
\frame{\tableofcontents[currentsection]}
\begin{frame}
  \frametitle{Conclusion}
  \begin{itemize}
  \item Data preparation
  \item Lasso
  \item Constrained regression
  \item Nonlinear model
  \item End-to-end optimization
  \end{itemize}    
\end{frame}

\begin{frame}
  \frametitle{Conclusion}
  More models:
  \begin{itemize}
  \item From machine learning: xgboost
  \item From statistics: GAM boosting
  \item Other objectives: transaction costs, drawdown, etc.
  \item Current portfolio as input
  \item Nonlinear with monotonicity constraints
  \item Portfolio optimization as a layer
  \end{itemize}
\end{frame}





\end{document}

\section{First section}
\subsection{First subsection}
\begin{frame}\frametitle{1.1.1 insert title} Insert text. \[ E = mc^2 \] \end{frame}
\begin{frame}\frametitle{1.1.2 insert title} Insert text.\end{frame}
\subsection{Second subsection}
\begin{frame}\frametitle{1.2.1 insert title} Insert text.\end{frame}
\section{Second section}

\begin{frame}
  \frametitle{Plot}
  \centering
<<PlotNormal>>=
library(quantmod)
getSymbols("^GSPC", auto.assign=TRUE)
op <- par( mar=c(2,3,1,1) )
plot( 
    index(GSPC), as.numeric( Ad(GSPC) ),
    type="l", lwd=3, log="y", las=1,
    xlab="", ylab=""
)    
par(op)
@ 
\end{frame}

\begin{frame}
  \frametitle{Plots}
  \centering
<<PlotTwo>>=
library(quantmod)
getSymbols("^GSPC", auto.assign=TRUE)
op <- par( mfrow=c(1,2), mar=c(2,3,4,1) )
plot( 
    index(GSPC), as.numeric( Ad(GSPC) ),
    type="l", lwd=3, log="y", las=1,
    xlab="", ylab="", main = "Price"
)    
plot( 
    index(GSPC), as.numeric( Vo(GSPC) ),
    type="h", lwd=1, las=1,
    xlab="", ylab="", main = "Volume"
)    
par(op)
@ 
\end{frame}

\begin{frame}
  \frametitle{Columns}
  %\footnotesize
  \scriptsize
  \begin{multicols}{2}
    \kant[4]
  \end{multicols}
\end{frame}

%%%%%%%%%%%%%%%%%%%%%%%%%%%%%%%%%%%%%%%%%%%%%%%%%%%%%%%%%%%%
%%
%% References
%%
%%%%%%%%%%%%%%%%%%%%%%%%%%%%%%%%%%%%%%%%%%%%%%%%%%%%%%%%%%%%

\clearpage
\nocite{*}                % Include the whole bibliography, even the entries we have not cited
%\bibliographystyle{amsalpha}
\bibliographystyle{these} % these.bst (a bibliography style similar to amsampha, with fewer numbers, and article summaries)
\bibliography{document}   % document.bib

%%%%%%%%%%%%%%%%%%%%%%%%%%%%%%%%%%%%%%%%%%%%%%%%%%%%%%%%%%%%
%%
%% Disclaimer
%%
%%%%%%%%%%%%%%%%%%%%%%%%%%%%%%%%%%%%%%%%%%%%%%%%%%%%%%%%%%%%


\begin{frame}[plain]
\disclaimersize
\parskip .5em
\textbf{Additional Information}

...
\end{frame}

\end{document}

